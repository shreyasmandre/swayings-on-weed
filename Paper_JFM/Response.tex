\documentclass[a4paper,12pt]{article}
\pagenumbering {roman}
\usepackage{amsmath}

\usepackage{dcolumn}% Align table columns on decimal point

\usepackage{graphicx}% Include figure files
\usepackage{dcolumn}% Align table columns on decimal point
\usepackage{bm}% bold math
\usepackage{ifthen}
\usepackage{amsthm} % Theorem Formatting
\usepackage{amssymb}	% Math symbols such as \mathbb
\usepackage{calrsfs}

\usepackage[above]{placeins}
\newcommand{\bv}{\mathbf{v}}
\newcommand{\bu}{\mathbf{u}}
\newcommand{\bx}{\mathbf{x}}
\newcommand{\bA}{\mathbf{A}}
\newcommand{\bU}{\mathbf{U}}
\newcommand{\grad}{\mathbf{\nabla}}
\newcommand{\del}{\partial}


\begin{document}
%\begin{abstract}
%aasfdakaldvgas;dfasdnk
%\end{abstract}


%\author{Ravi shanker singh }
%\begin{figure}
%\centerline{\includegraphics[scale=.5]{iitblogoHuge.pdf}}
%\end{figure}

 
%\centerline{\large{--------}}
%\vspace{1in}
%\centerline{\large {Department Of Physics}}
%\centerline{\Large{Brown University}}
%\centerline{\large{providence}}


We would like to take this opportunity to thank both the referees for reviewing our work and providing their valuable feedback. We are pleased that the referee find 
our research work to be interesting.


In this letter we have tried to address the concerns and questions raised by the referees.

\textbf{One of question raised by a referee is about constant velocity profile $U(y)=const$ with in the grass.} 
\newline
The origin of approximately constant velocity with in the grass 
can be understood by well known perturbation theory (Perturbation Methods by E.J. Hinch, Cambridge Texts in Applied Mathematics ). In case of sufficiently high grass density, the dominant balance of equation (3.1)
with in vegetation is between pressure gradient term $-\frac{dP}{dx}$ and the drag term experienced by the flow $-\rho C_N dNg U^2$ whereas the viscous term contribute very little 
to (3.1), so we can treat the viscous term $\mu U_{yy}$ with in the grass as a small perturbation term in equation (3.1) ( $U(y)$ is approximately constant as can be seen from experimental data as well from
numerical simulation of 3.1). Using the dominant balance of pressure gradient and drag term 
give rise to a constant velocity $U_g = \sqrt{\frac{-dp/dx}{\rho C_N d N_g}}$, more in depth detailed discussion about such perturbation techniques can be found in the the book by Hinch titles as ``perturbation
methods''
\newline
\textbf{ About the Origin of boundary layer $\delta$}
\newline
In order to match shear stress at the grass tip applied by the fluid above the grass, the flow with in the grass increases from an approximately constant velocity $U_g = \sqrt{\frac{-dp/dx}{\rho C_N d N_g}}$ in a region $\delta$ to a value so that both velocity and shear stress are continuous across the grass tip. Again for in-depth discussion about such problem can be found in the book of Hinch.   
\newline
\textbf{About the estimated value of $H/\delta$ from Figure-1}
\newline
I think reviewer have understood the case-I mentioned for Figure-1 as the one from 2002 paper ( Mixing layer and coherent structures in vegetated aquatic flows, \textit{J. Geophys. Res. 107} ), whereas we are referring to the case-I from the 2004 paper ( The limited growth of vegetated shear layers, \textit{Water Resource Research 40(7)} ). In this case the channel width is $41cm$, implying $H=20.5 cm$ hence $H/\delta=4.02$. We would also like to point out that we have used log scale on y-axis in Figure-2, and indeed $H/\delta \approx 5$ for all the experimental observation shown in Figure-2.
\newline
\textbf{Basis for eddy viscosity}
\newline
The constant eddy viscosity of 0.1 Pa s is chosen based on eddy viscosity profile shown in Figure-5 for lab scale experiments of Ghisalberti and Nepf (The limited growth of vegetated shear layers. \textit{Water Resources Research 40 (7), 2004} )





\end{document}
