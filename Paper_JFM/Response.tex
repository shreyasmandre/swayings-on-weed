\documentclass[a4paper,12pt]{article}
\pagenumbering {roman}
\usepackage{amsmath}

\usepackage{dcolumn}% Align table columns on decimal point

\usepackage{graphicx}% Include figure files
\usepackage{dcolumn}% Align table columns on decimal point
\usepackage{bm}% bold math
\usepackage{ifthen}
\usepackage{amsthm} % Theorem Formatting
\usepackage{amssymb}	% Math symbols such as \mathbb
\usepackage{calrsfs}

\usepackage[above]{placeins}
\newcommand{\bv}{\mathbf{v}}
\newcommand{\bu}{\mathbf{u}}
\newcommand{\bx}{\mathbf{x}}
\newcommand{\bA}{\mathbf{A}}
\newcommand{\bU}{\mathbf{U}}
\newcommand{\grad}{\mathbf{\nabla}}
\newcommand{\del}{\partial}


\begin{document}
%\begin{abstract}
%aasfdakaldvgas;dfasdnk
%\end{abstract}


%\author{Ravi shanker singh }
%\begin{figure}
%\centerline{\includegraphics[scale=.5]{iitblogoHuge.pdf}}
%\end{figure}

 
%\centerline{\large{--------}}
%\vspace{1in}
%\centerline{\large {Department Of Physics}}
%\centerline{\Large{Brown University}}
%\centerline{\large{providence}}


We would like to take this opportunity to thank both the referees for reviewing our work and providing their valuable feedback. We are pleased that the referee find 
our research work to be interesting.


In this letter we have tried to address the concerns and questions raised by the referees.

\textbf{One of question raised by a referee is about constant velocity profile $U(y)=const$ with in the grass.} 
\newline
The origin of approximately constant velocity with in the grass 
can be understood by well known perturbation theory (Perturbation Methods by E.J. Hinch, Cambridge Texts in Applied Mathematics ). In case of sufficiently high grass density, the dominant balance
with in vegetation is between pressure gradient term $-\frac{dP}{dx}$ and the drag term experienced by the flow $-\rho C_N dNg U^2$ whereas the viscous term contribute very little 
to (3.1), so we can treat the viscous term $\mu U_{yy}$ with in the grass as a small perturbation term in equation (3.1) ( $U(y)$ is approximately constant as can be seen from experimental data as well from
numerical simulation of 3.1). Using the dominant balance of pressure gradient and drag term 
give rise to a constant velocity $U_g = \sqrt{\frac{-dp/dx}{\rho C_N d N_g}}$, more in depth detailed discussion about such perturbation techniques can be found in the the book by Hinch titles as ``perturbation
methods''
\newline
\textbf{ About the Origin of boundary layer $\delta$}







\end{document}
