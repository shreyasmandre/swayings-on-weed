\documentclass[letterpaper,10pt]{article}
\pagenumbering {roman}
\usepackage{amsmath}

\usepackage{dcolumn}% Align table columns on decimal point

\usepackage{graphicx}% Include figure files
\usepackage{dcolumn}% Align table columns on decimal point
\usepackage{bm}% bold math
\usepackage{ifthen}
\usepackage{amsthm} % Theorem Formatting
\usepackage{amssymb}	% Math symbols such as \mathbb
\usepackage{calrsfs}

\usepackage[above]{placeins}
\newcommand{\bv}{\mathbf{v}}
\newcommand{\bu}{\mathbf{u}}
\newcommand{\bx}{\mathbf{x}}
\newcommand{\bA}{\mathbf{A}}
\newcommand{\bU}{\mathbf{U}}
\newcommand{\grad}{\mathbf{\nabla}}
\newcommand{\del}{\partial}


\begin{document}

\centerline{Response to Referee 1}
We thank the referee for reviewing our work and providing their valuable feedback. 
We are pleased that the referee finds our contribution interesting, and the relevance of our work to fundamental and practical fluid mechanics certain.
Here we respond to the questions raised and the feedback provided.

\begin{enumerate}
\item The referee says:
\textit{
The claims of the paper are somehow contradictory, in terms of the application to sea grass beds. The title focuses on “monami in a sea grass bed” whereas motion of the sea grass is excluded from the analysis.  For the conclusions of the paper to be applicable to sea grass motion two things would be needed : (a) a much larger discussion on the possible effects of the unstable modes on the motion of the grass, (b) a discussion on the potential role of sea grass flexibility in the instability of the coupled flow-grass system. As it is, the  title of the paper should rather  be “linear stability of the flow over a submerged sea grass bed”. In other terms, I am not sure that the results of the present paper (except the existence of a threshold) have consequences on the motion of the sea grass.
}

While we agree whole-heartedly with the spirit of the referee's comment, we disagree with some of th edetailed interpretation. more explanation. We have now modified the manucript to clarify this issue. The new manuscript now reads: ``blah blah''.

\item The referee says:
\textit{
The practical existence of a threshold flow condition for monami (or rather,  for flow instability ) to occur is doubtless.  As stated page 6, the inviscid KH instability approach does not yield any threshold, of course. The finding of a threshold by including dissipation and other aspects is no surprise. The main contribution of the paper seems then to be the second mode. This brings back to the issue of the relevance of this new mode to the motion of the sea grass.
}

Do we agree or disagree with the referee? Why? What is our argument and interpretation? How did we revise the manuscript in response to this comment?

\item The referee says:
\textit{
The comparison with the case of wind-canopy interactions is interesting but does not seem consistent with the analysis found in the reference [1]  below, where the instability seems to be more dominantly KH type in water  (when flexibility is assumed).  This would need some discussion, at least to point out the differences.
}

Similar response.

\end{enumerate}

\newpage
\centerline{Response to Referee 2}
We thank the referee for reviewing our work and providing their valuable feedback. 
We are glad that we were able to engage the referee's attention with our presentation.
Here we respond in detail to his/her comments.
\begin{enumerate}
\item The referee says:
\textit{
XXX
}

We say

\item The referee says:
\textit{
blah
}

Do we agree or disagree with the referee? Why? What is our argument and interpretation? How did we revise the manuscript in response to this comment?

\item The referee says:
\textit{
blah blah
}

Similar response.

\end{enumerate}

\newpage
\centerline{\textbf{Matched asymptotic solution}}
Present the complete solution here. Refer to this solution in our response.

The origin of approximately constant velocity with in the grass 
can be understood by well known perturbation theory (Perturbation Methods by E.J. Hinch, Cambridge Texts in Applied Mathematics ). In case of sufficiently high grass density, the dominant balance
with in vegetation is between pressure gradient term $-\frac{dP}{dx}$ and the drag term experienced by the flow $-\rho C_N dNg U^2$ whereas the viscous term contribute very little 
to (3.1), so we can treat the viscous term $\mu U_{yy}$ with in the grass as a small perturbation term in equation (3.1) ( $U(y)$ is approximately constant as can be seen from experimental data as well from
numerical simulation of 3.1). Using the dominant balance of pressure gradient and drag term 
give rise to a constant velocity $U_g = \sqrt{\frac{-dp/dx}{\rho C_N d N_g}}$, more in depth detailed discussion about such perturbation techniques can be found in the the book by Hinch titles as ``perturbation
methods''
\newline
\textbf{ About the Origin of boundary layer $\delta$}







\end{document}
