\documentclass[letterpaper,10pt]{article}
\pagenumbering {roman}
\usepackage{amsmath}

\usepackage{dcolumn}% Align table columns on decimal point

\usepackage{graphicx}% Include figure files
\usepackage{dcolumn}% Align table columns on decimal point
\usepackage{bm}% bold math
\usepackage{ifthen}
\usepackage{amsthm} % Theorem Formatting
\usepackage{amssymb}	% Math symbols such as \mathbb
\usepackage{calrsfs}

\usepackage[above]{placeins}
\newcommand{\bv}{\mathbf{v}}
\newcommand{\bu}{\mathbf{u}}
\newcommand{\bx}{\mathbf{x}}
\newcommand{\bA}{\mathbf{A}}
\newcommand{\bU}{\mathbf{U}}
\newcommand{\grad}{\mathbf{\nabla}}
\newcommand{\del}{\partial}

%\newcommand{\bx}{{\boldsymbol{\hat{x}}}}
\newcommand{\bn}{{\boldsymbol{\hat{n}}}}
\newcommand{\bt}{{\boldsymbol{\hat{t}}}}
%\newcommand{\bu}{\mathbf{u}}
%\newcommand{\grad}{\mathbf{\nabla}}
%\newcommand{\del}{\partial}
\newcommand{\hg}{h_g}
%\renewcommand{\Rey}{{R}}
\newcommand{\Ndg}{\tilde{N}_g}
\newcommand{\monami}{\textit{monami}}
\newcommand{\ubl}{U_\text{bl}}
\newcommand{\words}[1]{\textbf{(#1)~}}
\DeclareMathOperator{\sech}{sech}
\DeclareMathOperator{\csch}{csch}


\begin{document}

\centerline{Response to Referee 1}
We thank the referee for reviewing our work and providing their valuable feedback. 
We are pleased that the referee finds our contribution interesting, and the relevance of our work to fundamental and practical fluid mechanics certain.
Here we respond to the questions raised and the feedback provided.

\begin{enumerate}
\item The referee says:
\textit{
The claims of the paper are somehow contradictory, in terms of the application to sea grass beds. The title focuses on “monami in a sea grass bed” whereas motion of the sea grass is excluded from the analysis.  For the conclusions of the paper to be applicable to sea grass motion two things would be needed : (a) a much larger discussion on the possible effects of the unstable modes on the motion of the grass, (b) a discussion on the potential role of sea grass flexibility in the instability of the coupled flow-grass system. As it is, the  title of the paper should rather  be “linear stability of the flow over a submerged sea grass bed”. In other terms, I am not sure that the results of the present paper (except the existence of a threshold) have consequences on the motion of the sea grass.
}

While we agree whole-heartedly with the spirit of the referee's comment, we disagree with some of th detailed interpretation. more explanation. We have now modified the manuscript to clarify this issue. The new manuscript now reads: ``blah blah''.

\item The referee says:
\textit{
The practical existence of a threshold flow condition for monami (or rather,  for flow instability ) to occur is doubtless.  As stated page 6, the inviscid KH instability approach does not yield any threshold, of course. The finding of a threshold by including dissipation and other aspects is no surprise. The main contribution of the paper seems then to be the second mode. This brings back to the issue of the relevance of this new mode to the motion of the sea grass.
}

While here we agree with the referee that prediction of threshold flow condition and existence of a mode whose asymptotic behavior is very different from otherwise assumed behavior due to KH instability is one of the major finding of our work. We would also like to point out that without that our model successfully predicts threshold Reynolds number and  waving frequency  as well without making any adhoc assumption about velocity profile. 

Since all the current experimental data fall into a regime where unstable region in R-k space (Figure-4) has not split into two, we are unable to say if the flow instability is due to Mode1 or Mode2. 

\item The referee says:
\textit{
The comparison with the case of wind-canopy interactions is interesting but does not seem consistent with the analysis found in the reference [1]  below, where the instability seems to be more dominantly KH type in water  (when flexibility is assumed).  This would need some discussion, at least to point out the differences.
}

It is true that inclusion of flexibility can give interesting results and we thank referee for drawing our attention to the interesting work of Gosselin $\&$ E. de Langre, where their model predicts enhancement of growth rate in the lock-in range due to less dissipation because of in phase movement of the flow and canopy. While in our present analysis we haven't taken into account the flexibility of plant as marine plants are flexible as compared to the terrestrial plants, we do expect deviation from the current prediction when the frequency of flow is similar to that of natural frequency of plant.  

\end{enumerate}

\newpage
\centerline{Response to Referee 2}
We thank the referee for reviewing our work and providing their valuable feedback. 
We are glad that we were able to engage the referee's attention with our presentation.
Here we respond in detail to his/her comments.
\begin{enumerate}
\item The referee says:
\textit{
XXX
}

We say

\item The referee says:
\textit{p.5 +10,11, Although experimentally observed wavelengths are not available, they do arise from the analysis of presumably, so some comment about the values and wheater they seem physically realistic (and/or how they compare with the scales of typical motions in the turbulent flow above the canopy which are believed to be linked with monami) would be instructive.
}

We thank referee to point out for the usefulness to mention the wavelengths of the dominant mode. We have found that for the lab scale experiments, the wavelength of dominant mode is comparable to half channel width (H). 

\item The referee says:
\textit{
The final sentence, if I understand this correctly, implies that all the
available data suggests that only mode 1 (i.e. only the left hand panels in
fig.4) is actually relevant to real situations. On the other hand, the
statement at the bottom on p.7 (mode 2 is distinct from KH), and the
conclusions, seem together to imply that the observed monami are mode 2.
I became confused at this point and would welcome greater clarity.
And if mode 2 is not observed in experiments, the extended analysis of its
character is perhaps not particularly relevant and should be shortened.
}

It is true that all experimental data we have found corresponds to a vegetation density for which the unstable region in the R-k space has not split into two, so we are unable to say if flow instability in the lab scale experiments of Ghisalberti and Nepf are due to Mode1 or Mode 2. While currently Mode 2 is explicitly obseved we believe that its extended analysis would be crucial for someone who would like to test the prediction by designing suitable experiment.

\item The referee says:
\textit{
P.4 +8:  
It's not clear how a boundary layer thickness can be equated to a velocity
ratio! This comment also appears in the caption of fig.1.
Furthermore, the caption states that the profile is from Ghislaberti Nepf
- case I. This case has H=12.3 cm and hg =9 cm and the caption says that $\delta$
= 5.02 cm. This implies an $H/\delta$ rather lower than any of the experimental
data shown in fig.2 (left), which seems inconsistent.
}

Since the shear stress $U_y$ is continuous across the grass tip we can equate the estimate of shear stress just above $(h_g^+$ the grass with that of just below the grass $h_g^{-}$. 
Above the grass tip the base flow is a parabolic velocity profile and the estimate of $U_y$ at the grass tip is $U_0/H$, whereas below the grass tip the shear stress can be estimated to be $U_{bl}/\delta$, equating the shear stress above the grass and below the grass gives us $\delta/H = U_{bl}/U_0$.
I think reviewer have understood the case-I mentioned for Figure-1 as the one from 2002 paper ( Mixing layer and coherent structures in vegetated aquatic flows, \textit{J. Geophys. Res. 107} ), whereas we are referring to the case-I from the 2004 paper ( The limited growth of vegetated shear layers, \textit{Water Resource Research 40(7)} ). In this case the channel width is $41cm$, implying $H=20.5 cm$ hence $H/\delta=4.02$. We would also like to point out that we have used log scale on y-axis in Figure-2, and indeed $H/\delta \approx 5$ for all the experimental observation shown in Figure-2.


\item The referee says:

\textit{  Fig 2 $\& 3$:
In both captions a value for the eddy viscosity is given. Was there any
physical basis for this value? If not, how was it chosen? And how critical
are the comparisons between theory and experiment to its value?
}

The constant eddy viscosity of 0.1 Pa s is chosen based on eddy viscosity profile shown in Figure-5 for lab scale experiments of Ghisalberti and Nepf (The limited growth of vegetated shear layers. \textit{Water Resources Research 40 (7), 2004} )



\end{enumerate}

\newpage
\centerline{\textbf{Matched asymptotic solution}}

Here we present asymptotic solution of equation 3.1, which should be helpful in clarifying many questions referee have related to the boundary layer near the grass tip. For simplicity of calculation in this particular calculation we assume grass extend from $y=0$ to $y=H$ ie submergance ratio is 0.5. 
Using $ U_0 = (-dP/dx)H^2/\mu$ as velocity scale, $H$ as length scale, we rescale velocity as $U= U_0 \bar{U}$ and $y = H \bar{y}$, the equation 3.1 can be transformed into  
\begin{equation}
\begin{split}
 \bar{U}_{\bar{y}\bar{y}}+1 - \frac{\rho U_0 H}{\mu} C_N d N_g \bar{U}^2 &=0 \\
 \bar{U}_{\bar{y}\bar{y}}+1 - R \Ndg \bar{U}^2=0
\end{split}
\end{equation}
We will drop the overbar and for further analysis for simplicity and define drag parameter as $D_g = R\Ndg$
\begin{equation}
\begin{split}
 {U}_{{y}{y}}+1 - D_g{U}^2 &=0
\end{split}
\end{equation}
where $D_g = 0$ for $0<y<=1$, grass tip is at $y=0$ and the bottom surface is at $y=-1$, we use the zero shear boundary condition for on both the surface as mentioned in the manuscript.

\textbf{Asymptotic Solution of 3.1 below the grass}

We observe that $U = \sqrt{(1/D_g)}$ is a stationary point of above equation, where both $U_{yy}=0$ and $U_y =0 $. By multiplying with $U_y$ and integrating once will give us following equation
\begin{equation}
\begin{split}
 \frac{1}{2} \left( \frac{dU}{dy} \right)^2 +U - \frac{1}{3} D_g U^3 + C = 0
\end{split}
\end{equation}
Where C is a constant of integration, whose can be determined by observing that as $U$ approaches $1/sqrt{D_g}$, $U_y$ approaches zero. Applying this condition gives $C = -\frac{2}{3\sqrt{D_g}}$. Using the value of $C$, we can further write above equation as 
\begin{equation}
\begin{split}
 \frac{dU}{dy} = \sqrt{\frac{2}{3}D_g U^3+\frac{4}{3\sqrt{D_g}}-2U }
\end{split}
\end{equation}
We further use the scaling $U=\frac{1}{\sqrt{D_g}} u $, in the above equation, which simplified into 
\begin{equation}
\begin{split}
 \frac{1}{\sqrt{D_g}}\frac{du}{dy} &= \frac{1}{D_g^{1/4}}\sqrt{\frac{2}{3}} \sqrt{ u^3+2-3u } \\
 \frac{du}{dy} &= {D_g^{1/4}}\sqrt{\frac{2}{3}} \sqrt{ u^3+2-3u } \\
 \frac{du}{(u-1)\sqrt{u+2} } &= {D_g^{1/4}}\sqrt{\frac{2}{3}} dy
\end{split}
\end{equation}
Above equation can be integrated which provides solution in the region of grass
\begin{equation}
\begin{split}
u &= 3 \coth \left(\frac{C_1-y D_g^{1/4}}{\sqrt{2}}  \right)^2-2 \\
U &= \frac{1}{\sqrt{D_g}} \left( 3 \coth^2 \left(\frac{C_1-y D_g^{1/4}}{\sqrt{2}}  \right)-2    \right) \hspace{1cm} \text{for $-1<=y<=0$}
\label{under_grass_sol}
\end{split}
\end{equation}
Where $C_1$ is constant of integration, which can be found by matching the shear stress applied by the flow above the grass at the tip of the grass.

\textbf{Solution above the grass}

Above the grass tip the non-dimenlize equation form of equation 3.1 translates in to.
\begin{equation}
 U_{yy}+1=0
\end{equation}
which can be integrated together with the zero shear boundary condition at top to provided
\begin{equation}
 U(y) = y-y^2/2+C_2 \hspace{1cm} \text{for $0<y<1$}
 \label{above_grass_sol}
\end{equation}
Where $C_2$ is constant of integration.

\textbf{Matching the solution obtained by \eqref{under_grass_sol} and \eqref{above_grass_sol} }

The value of $C_1$ and $C_2$ can be determined by matching the solution \eqref{under_grass_sol} and \eqref{above_grass_sol} at the grass tip.
Matching stress stress at the grass tip results into.
\begin{equation}
\frac{3\sqrt{2}}{D_g^{1/4}} \coth(C_1/\sqrt{2}) \csch^2(C_1/\sqrt{2}) = 1;
\label{C1_eq}
\end{equation}
Above equation can be solved through some numerical method like bisection etc, here we will do an asymptotic expansion of terms in \eqref{C1_eq} to get an estimate of dependence of $C_1$ on the drag parameter.
\begin{equation}
\begin{split}
\frac{3\sqrt{2}}{D_g^{1/4}} \frac{\sqrt{2}}{C_1} \frac{2}{C_1^2} &= 1 \\
C_1 = \frac{(12)^{1/3}}{D_g^{1/12}}
\end{split}
\end{equation}
Through constant $C_1$ a natural length scale $\delta$ arises when $\delta D_g^{1.4}$ balances $C_1$ in equation \eqref{under_grass_sol}, which gives $\delta \sim D_g^{-1/3}$ or $\delta \sim (R\Ndg)^{-1/3}$. We can also obtain scale of velocity in the boundary layer by substituting $y=0$ in the \eqref{under_grass_sol} and using asymptotic expansion of $\coth(x)$, doing this expansion leads to 
\begin{equation}
\begin{split}
U(y)-U_{tip} &\sim \frac{1}{\sqrt{D_g}}\frac{1}{C_1^2}\\
U(y)-U_{tip} &\sim D_g^{-1/3}\\
U(y)-U_{tip} &\sim (R\Ndg)^{-1/3}
\end{split}
\end{equation}
near the grass tip. More detailed analysis of similar kind can be understood by well known perturbation theory (Perturbation Methods by E.J. Hinch, Cambridge Texts in Applied Mathematics ).
\newline
\textbf{ About the Origin of boundary layer $\delta$}

\centerline{Ravi's new edits}

In order to match shear stress at the grass tip applied by the fluid above the grass, the flow with in the grass increases from an approximately constant velocity $U_g = \sqrt{\frac{-dp/dx}{\rho C_N d N_g}}$ in a region $\delta$ to a value so that both velocity and shear stress are continuous across the grass tip. Again for in-depth discussion about such problem can be found in the book of Hinch.   
\newline
\textbf{About the estimated value of $H/\delta$ from Figure-1}
\newline
I think reviewer have understood the case-I mentioned for Figure-1 as the one from 2002 paper ( Mixing layer and coherent structures in vegetated aquatic flows, \textit{J. Geophys. Res. 107} ), whereas we are referring to the case-I from the 2004 paper ( The limited growth of vegetated shear layers, \textit{Water Resource Research 40(7)} ). In this case the channel width is $41cm$, implying $H=20.5 cm$ hence $H/\delta=4.02$. We would also like to point out that we have used log scale on y-axis in Figure-2, and indeed $H/\delta \approx 5$ for all the experimental observation shown in Figure-2.
\newline
\textbf{Basis for eddy viscosity}
\newline
The constant eddy viscosity of 0.1 Pa s is chosen based on eddy viscosity profile shown in Figure-5 for lab scale experiments of Ghisalberti and Nepf (The limited growth of vegetated shear layers. \textit{Water Resources Research 40 (7), 2004} )





\end{document}
