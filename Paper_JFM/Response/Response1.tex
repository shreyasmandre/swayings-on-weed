\documentclass[letterpaper,10pt]{article}
% \documentclass{jfm}
\pagenumbering {roman}
\usepackage{amsmath}

\usepackage{dcolumn}% Align table columns on decimal point

\usepackage{natbib}
\usepackage{graphicx}% Include figure files
\usepackage{dcolumn}% Align table columns on decimal point
% \usepackage{bm}% bold math
% \usepackage{amssymb}	% Math symbols such as \mathbb
\usepackage{calrsfs}
\usepackage{fullpage}
\usepackage{color}
\usepackage[above]{placeins}
\newcommand{\bv}{\mathbf{v}}
\newcommand{\bu}{\mathbf{u}}
\newcommand{\bx}{\mathbf{x}}
\newcommand{\bA}{\mathbf{A}}
\newcommand{\bU}{\mathbf{U}}
\newcommand{\grad}{\mathbf{\nabla}}
\newcommand{\del}{\partial}

%\newcommand{\bx}{{\boldsymbol{\hat{x}}}}
\newcommand{\bn}{{\boldsymbol{\hat{n}}}}
\newcommand{\bt}{{\boldsymbol{\hat{t}}}}
%\newcommand{\bu}{\mathbf{u}}
%\newcommand{\grad}{\mathbf{\nabla}}
%\newcommand{\del}{\partial}
\newcommand{\hg}{h_g}
\newcommand{\Rey}{{R}}
\newcommand{\Ndg}{\tilde{N}_g}
\newcommand{\monami}{\textit{monami}}
\newcommand{\ubl}{U_\text{bl}}
\newcommand{\words}[1]{\textbf{(#1)~}}
\DeclareMathOperator{\sech}{sech}
\DeclareMathOperator{\csch}{csch}
\newcommand{\revise}[1]{{\color{blue} {#1}}}

\begin{document}

\centerline{Response to Referee 1}
We thank the referee for reviewing our work and providing their valuable feedback. 
We are pleased that the referee finds our contribution interesting, and the relevance of our work to fundamental and practical fluid mechanics certain.
Here we respond to the questions raised and the feedback provided.

\begin{enumerate}
\item The referee says:
\textit{
The authors claim that their approach yields not only the classical Kelvin-Helmholtz instability modes know[sic] for these systems, but also a new type of instability mode. The later is thought to be present in practical cases.
}

The referee did understand the gist of our conclusion, but we would like to take this opportunity to clarify further our position on this matter. In summary, we have refrained from making a determination about which mode is observed in experiments and the field. It is so because all experimental and field data available to us exists in a parameter regime in which the two modes cannot be distinguished. 

We remind the referee that the two modes bifurcate above a critical vegetation density from what appears to be a single mode. Below this value, the physical processes underlying the two unstable mode are both active, we have difficulty distinguishing between them. 

Our approach could be stated in terms of two questions: (i) do experiments support our mathematical formulation, and (ii) of the two modes, which one is observed in experiments/field? The answer to the first question is a likely yes. The threshold condition and the waving frequency both are in good agreement with the results of our analysis. The question of which mode is physically observed is more subtle, because it is a question of fluid dynamic mechanism. We think there is a danger of using superficial flow features to deduce the underlying mechanism. Therefore, the approach we took was to distinguish between the modes and the underlying fluid dynamics using the dense grass asymptotic limit. A concise and complete presentation of the difference is presented in Table 1 of the manuscript. One conclusion that can be drawn from our analysis is a challenge to the traditional view that the observed instability is Kelvin-Helmholtz. Instead, we conclude that that there are two modes of instability, and which one 
is predominantly active in the experiments and the field is still an open question and should be subject of future investigations. 

To clarify this discussion in the manuscript, we now say after \S 4:

\noindent
\revise{All experimental data we have found corresponds to a vegetation density for which the unstable region in the $\Rey-k$ space has not split into two, so we are unable to determine if flow instability in the lab scale experiments \citep{Ghisal02} are due to Mode 1 or Mode 2.}

and in the conclusion:

\noindent
\revise{
We are unable to determine based on observations, and therefore have refrained from identifying, which mode is observed in experiments and it still remains a subtle question and subject of future investigation. Since the two modes merge for the experimental parameters, KH may not be assumed to underlie \monami.
}

\item The referee says:
\textit{
The claims of the paper are somehow contradictory, in terms of the application to sea grass beds. The title focuses on “monami in a sea grass bed” whereas motion of the sea grass is excluded from the analysis.  For the conclusions of the paper to be applicable to sea grass motion two things would be needed : (a) a much larger discussion on the possible effects of the unstable modes on the motion of the grass, (b) a discussion on the potential role of sea grass flexibility in the instability of the coupled flow-grass system. As it is, the  title of the paper should rather  be “linear stability of the flow over a submerged sea grass bed”. In other terms, I am not sure that the results of the present paper (except the existence of a threshold) have consequences on the motion of the sea grass.
}

While we agree whole-heartedly with the spirit of the referee's comment, we disagree with some of the detailed interpretation. 
Experimental observations have revealed that the flow structures underlying \monami~ persist when the flexible grass mimics are replaced by rigid woden dowels.
These observations are interpreted as support for the origin of the \monami~ being a hydrodynamic in nature.
In this simple picture, the flexible grass simply deforms in response to flow structures.
In reality, the flexibility of the grass does modify the details of the instability.
Previous investigations have identified this modification to be in the form of frequency-locking of the hydro-elastic oscillations with the hydrodynamic instability frequency, with different strengths depending on the canopy being terrestrial \citep{Delangre04,Delangre06} or aquatic \cite{Gosselin2009}.
However, the motion of the vegetation is not \textit{essential} to the existence of the instability, and therefore a minimal model is justified in neglecting it.
This approach in simplification is similar in spirit to, for example, the one employed by \cite{Gosselin2009}, in which they determine the frequency of the coupled fluid-vegetation oscillations in the absence of the flow, and then use the frequency in a dynamic model in the presence of the flow but assuming the mode shape \textit{ad hoc} instead of solving for it self-consistently. 
Since the precise mode shape is not an essential feature for the phenomenon of frequency locking, such a substitution yields simplification without reducing the utility of the analysis.

However, we also recognize the referee's concern with the neglect of the vegetation flexibility.
In particular, we have presented a mathematical model for a rigid vegetation bed, and therefore it will be useful to determine boundaries in the parameter space where our results will be quantitatively applicable, instead of merely being qualitatively useful.
To alleviate this concern, we present a back-of-the-envelope analysis determining the criteria for neglecting the vegetation flexibility.

We have now added the following analysis to the discussion:

\revise{
We now test the assumption of an undeformable grass bed due to the dominant restoring force of buoyancy, using the criteria that the buoyancy time scale be much shorter than the hydrodynamic time scale $H/U_0$.
For a common seagrass, \textit{Zostera Marina}, the relative density difference is $\Delta \rho /\rho \approx 0.25$, the volume fraction is $V_f \approx 0.1$ and $H=1$ m \citep{Fonseca98}, yielding the buoyancy time scale to be $\sqrt{\rho H/V_f \Delta \rho g} \approx 2$ s.
The hydrodynamic time scale assuming $U_0 \approx 0.1$ m/s is 10 s, and therefore longer than the buoyancy time scale.
We have neither accounted for the pre-factors appearing in the scaling argument, or considered cases when the time-scale separation is not so evident. 
Accounting for these factors can lead to further interesting behavior \citep{Delangre06,Gosselin2009}, and motivates further investigation. 
}

\item The referee says:
\textit{
The practical existence of a threshold flow condition for monami (or rather,  for flow instability ) to occur is doubtless.  As stated [on] page 6, the inviscid KH instability approach does not yield any threshold, of course. The finding of a threshold by including dissipation and other aspects is no surprise. The main contribution of the paper seems then to be the second mode. This brings back to the issue of the relevance of this new mode to the motion of the sea grass.
}

We agree partly with the referee. The existence of a threshold, in our opinion also, is doubtless. 
But it is not obvious to deduce that including some aspect of dissipation would yield that threshold. 
We are sure the referee knows that sometimes dissipative effects can be destabilizing \citep{Krechetnikov2007}. 
In our case, adding viscous (turbulent) dissipation alone does not lead to a threshold in Reynolds number for the shear instability of a tanh profile. 
Furthermore, adding vegetation-induced drag adds another parameter to the problem, and its interaction with the turbulent dissipation in establishing a threshold is not trivial. 
In fact, we find very non-trivial dependency of the threshold criteria on both the turbulent dissipation parameter ($\Rey$) and the vegetation drag parameter ($\Ndg$). 
Therefore, following our analysis, we disagree with the referee that finding the threshold  by including dissipation or other effects is expected. 
Only an analysis such as ours can conclusively establish the threshold.

We also agree with the referee that a significant contribution of our analysis is the Mode 2 of instability.
As explained in item 1 of this response, it is not clear which mode of instability underlies \monami.
Since the existence of the two modes was unknown before our analysis, Mode 2 as a possible candidate for \monami~ was never considered. 
We are unable to determine which of the two modes predominantly underlies \monami.
Therefore, Mode 2 remains a possible candidate for the synchronous waving of aquatic vegetation.

\item The referee says:
\textit{
The comparison with the case of wind-canopy interactions is interesting but does not seem consistent with the analysis found in the reference [1]  below, where the instability seems to be more dominantly KH type in water  (when flexibility is assumed).  This would need some discussion, at least to point out the differences.
}

It is true that inclusion of flexibility can give interesting results and we thank referee for drawing our attention to the interesting work of \cite{Gosselin2009}, where their model correlates enhancement of growth rate in the frequency lock-in range. 
While in our present analysis we have not taken into account the flexibility of plant, marine plants are flexible as compared to the terrestrial plants.
Buoyancy as a restoring force for marine vegetation dominates over the restoring action of the vegetation bending stiffness.
For this reason, we do expect deviation from the current prediction when the frequency of flow instability is comparable to that of natural (buoyancy) frequency of grass bed.  
We address this issue in item 2 of this response.

In addition, we would like to point out that the instability mode found by \cite{Gosselin2009} is assumed to be KH, but not carefully demonstrated to be so.
Thus, we interpret the referee's comment as saying that the flexibility enhances the instability of whichever mode is observed by \cite{Gosselin2009}. 
This mode could very well be Mode 2 that we found. 
Further investigation is needed before this question can be conclusively answered.
Especially, cleverly designed experiments with observations to distinguish between the two modes are needed in the future.

\end{enumerate}

\bibliography{Grass}{}
\bibliographystyle{jfm}
% 
% \newpage
% \textbf{ About the Origin of boundary layer $\delta$}
% 
% \centerline{Ravi's new edits}
% 
% In order to match shear stress at the grass tip applied by the fluid above the grass, the flow with in the grass increases from an approximately constant velocity $U_g = \sqrt{\frac{-dp/dx}{\rho C_N d N_g}}$ in a region $\delta$ to a value so that both velocity and shear stress are continuous across the grass tip. Again for in-depth discussion about such problem can be found in the book of Hinch.   
% 
% \textbf{About the estimated value of $H/\delta$ from Figure-1}
% 
% I think reviewer have understood the case-I mentioned for Figure-1 as the one from 2002 paper ( Mixing layer and coherent structures in vegetated aquatic flows, \textit{J. Geophys. Res. 107} ), whereas we are referring to the case-I from the 2004 paper ( The limited growth of vegetated shear layers, \textit{Water Resource Research 40(7)} ). In this case the channel width is $41cm$, implying $H=20.5 cm$ hence $H/\delta=4.02$. We would also like to point out that we have used log scale on y-axis in Figure-2, and indeed $H/\delta \approx 5$ for all the experimental observation shown in Figure-2.
% 
% \textbf{Basis for eddy viscosity}
% 
% The constant eddy viscosity of 0.1 Pa s is chosen based on eddy viscosity profile shown in Figure-5 for lab scale experiments of Ghisalberti and Nepf (The limited growth of vegetated shear layers. \textit{Water Resources Research 40 (7), 2004} )


\end{document}
