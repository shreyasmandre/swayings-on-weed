\documentclass[10pt]{letter}
\usepackage{fullpage}
\usepackage{hyperref}
\usepackage{palatino}
\name{Shreyas Mandre}
\date{\today}
% \address{
% Shreyas Mandre \\
%182 Hope Street \\
%Box D \\
%Providence RI 02912
%}

\telephone{\url{shreyas_mandre@brown.edu} \quad +1 401 863 2602}
\signature{Shreyas Mandre \\ Assistant Professor \\ School of Engineering, Brown University}
\pagestyle{empty}
\begin{document}
\begin{letter}{
Editor \\
Physical Review Letters \\
1 Research Road, \\
Ridge, NY 11961-2701 
}
\opening{Dear Editor,}

Please consider a manuscript entitled, ``{\it Monami} as an oscillatory hydrodynamic instability in a submerged sea grass bed'' by Ravi Singh, Amala Mahadevan, and myself for publication in  Physical Review Letters. This manuscript addresses the synchronous coherent oscillations of aquatic vegetation in a steady flow, known as {\it monami}. The phenomenon is analogous to the more commonly observed synchronous waving of terrestrial grasses in the wind. In the manuscript, we present a mathematical model for the onset of these oscillations.

Our work examines the instability at the interface of a submerged grass bed and overlying fluid in the presence of a unidirectional flow.  The impedance of the flow by the grass sets up a shear layer above the grass.   It has been previously postulated from experiments that this shear layer is susceptible to  Kelvin-Helmholtz instability, which drives the waving of the grass.  Here, we challenge this notion by  showing that the presence of drag due to the grass generates a different instability mechanism, distinct from Kelvin-Helmholtz instability. This new instability has characteristics consistent with the experimental observations, and our theory makes verifiable predictions about the threshold velocity for its onset.

Coherent motion of seagrass meadows enhances the transport of nutrients, gametes, larvae, and other microscopic life forms within the habitat they form. This transport in turn influences the seagrass ecosystem and regulates the biomass in it. Although seagrasses occupy less than 0.05\% of the ocean area, they contribute directly to about 15\% of the total biomass production rate in the ocean, provide habitat to thousands of other species, sequester carbon, recycle nutrients, and stabilize sediments. Due to their effectiveness in performing these functions, seagrasses are considered some of the world's most valuable resources. An improved understanding of the processes performed very effectively by the seagrasses is critical in managing the planet's seagrass meadows. The synchronous motion of the meadow is one such process allowing the submerged seagrass meadow to exchange matter with the overlying fluid. Our manuscript is one of the few theoretical treatments on the physical mechanism behind these oscillations. 

%Previous experiments showed the existence of a shear layer between the submerged grass and overlying fluid that was conjectured to be susceptible to  Kelvin-Helmholtz instability.
%
%Our analysis shows that the existing explanation for these oscillations is incorrect. The existing explanation invokes the existence of a shear layer near the top of the grass bed, which is established due to the drag exerted by the grass of the flow. This shear layer is assumed to be unstable to the Kelvin-Helmholtz instability; the vortices that form as a result of this instability then drive the synchronous oscillations of the grass blades. We test this explanation mathematically in our manuscript, using a self-consistent linear stability analysis of the flow. We demonstrate that indeed the steady flow is hydrodynamically unstable and the instability mechanism, although superficially similar to the Kelvin-Helmholtz instability, is in fact distinct from it. The results from our model are supported by qualitative observations in the field and quantitative lab experiments reported in the literature.

We believe that the topic will be of interest because of its pertinence to the commonplace terrestrial analogy and the importance of the seagrass ecosystems. We have endeavored to write it in a form that is accessible to the general physicist. 

Should you have any questions, please do not hesitate to contact me.

\closing{Thanking you, \\Yours Sincerely,}

\end{letter}
\end{document}
