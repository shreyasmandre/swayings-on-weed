\documentclass[10pt]{letter}
\usepackage{fullpage}
\usepackage{hyperref}
\usepackage{palatino}
\name{Shreyas Mandre}
\date{\today}
% \address{
% Shreyas Mandre \\
% 29 Oxford Street \#328 \\
% Pierce Hall Harvard University \\
% Cambridge MA 02138 USA
% }

\telephone{\url{shreyasmandre@gmail.com} \quad +1 401 863 2602}
\signature{Shreyas Mandre}
\pagestyle{empty}
\begin{document}
\begin{letter}{
Editor \\
Physical Review Letters \\
1 Research Road, \\
Ridge, NY 11961-2701 
}
\opening{Dear Editor,}

Please consider a manuscript entitled, ``{\it Monami} as an oscillatory hydrodynamic instability in a submerged sea grass bed'' by Ravi Singh, Amala Mahadevan, and myself for publication in the Physical Review Letters. The subject of this manuscript is the synchronous coherent oscillations of marine grass, known as {\it monami} when subject to steady flow. These oscillations are similar to but distinct from the analogous phenomenon on terrestrial grass subject to strong wind. In the manuscript, we present a mathematical model for the onset of these oscillations.

Coherent motion of the grass blades enhances the transport of nutrients, gametes, larvae, and other microscopic life forms within the grass bed. This transport in turn dictates the seagrass ecosystem and regulates the biomass in it. Although seagrasses occupy less than 0.05\% of the ocean area, they contribute directly to about 15\% of the total biomass production rate in the ocean. Further, the seagrass meadows engineer a habitat supporting thousands of other species, sequester carbon dioxide, recycle nutrients, and stabilize sediments. Due to their effectiveness in performing these functions, seagrasses are considered some of the world's most valuable resources. An improved understanding of the processes performed very effectively by the seagrasses is critical in managing the planet's seagrass meadows. The synchronous motion of the meadow is one such process allowing the submerged seagrass meadow to ``breathe''. Our manuscript is one of the first theoretical treatments on the physical mechanism behind these oscillations. 

Our analysis shows that the existing explanation for these oscillations is incorrect. The existing explanation invokes the existence of a shear layer near the top of the grass bed, which is established due to the drag exerted by the grass of the flow. This shear layer is assumed to be unstable to the Kelvin-Helmholtz instability; the vortices that form as a result of this instability then drive the synchronous oscillations of the grass blades. We test this explanation mathematically in our manuscript, using a self-consistent linear stability analysis of the flow. We demonstrate that indeed the steady flow is hydrodynamically unstable and the instability mechanism, although superficially similar to the Kelvin-Helmholtz instability, is in fact distinct from it. The results from our model are supported by qualitative observations in the field and quantitative lab experiments reported in the literature.

We believe that the topic will be of interest to a wide audience, not only because of the relation to the commonplace terrestrial analogy, but also because of the importance of the seagrass ecosystems. We have endeavored to write it in a form that is accessible to the general physicist. 

Should you have any questions, please do not hesitate to contact me.

\closing{Thanking you, \\Yours Sincerely,}

\end{letter}
\end{document}
